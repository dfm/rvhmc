\documentclass[modern]{aastex62}

\pdfoutput=1

\usepackage{lmodern}
\usepackage{microtype}
\usepackage{url}
\usepackage{amsmath}
\usepackage{amssymb}
\usepackage{natbib}
\usepackage{multirow}
\usepackage{graphicx}
\bibliographystyle{aasjournal}

% Matrix fix:
% http://tex.stackexchange.com/questions/317824/letter-c-appearing-inside-pmatrix-environment-with-aastex
\makeatletter
\def\env@matrix{\hskip -\arraycolsep\let\@ifnextchar\new@ifnextchar\array{*{\c@MaxMatrixCols}c}}
\makeatother

% Column spacing in matrix
% http://tex.stackexchange.com/questions/275725/adjusting-separation-between-matrix-entries
\setlength\arraycolsep{25pt}

% ------------------ %
% end of AASTeX mods %
% ------------------ %

% Projects:
\newcommand{\project}[1]{\textsf{#1}}
\newcommand{\kepler}{\project{Kepler}}
\newcommand{\ktwo}{\project{K2}}

% references to text content
\newcommand{\documentname}{\textsl{Note}}
\newcommand{\figureref}[1]{\ref{fig:#1}}
\newcommand{\Figure}[1]{Figure~\figureref{#1}}
\newcommand{\figurelabel}[1]{\label{fig:#1}}
\renewcommand{\eqref}[1]{\ref{eq:#1}}
\newcommand{\Eq}[1]{Equation~(\eqref{#1})}
\newcommand{\eq}[1]{\Eq{#1}}
\newcommand{\eqalt}[1]{Equation~\eqref{#1}}
\newcommand{\eqlabel}[1]{\label{eq:#1}}

% TODOs
\newcommand{\todo}[3]{{\color{#2}\emph{#1}: #3}}
\newcommand{\dfmtodo}[1]{\todo{DFM}{red}{#1}}
\newcommand{\alltodo}[1]{\todo{TEAM}{red}{#1}}
\newcommand{\citeme}{{\color{red}(citation needed)}}

% math
\newcommand{\T}{\ensuremath{\mathrm{T}}}
\newcommand{\dd}{\ensuremath{ \mathrm{d}}}
\newcommand{\unit}[1]{{\ensuremath{ \mathrm{#1}}}}
\newcommand{\bvec}[1]{{\ensuremath{\boldsymbol{#1}}}}
\newcommand{\Gaussian}[3]{\ensuremath{\frac{1}{|2\pi #2|^\frac{1}{2}}
            \exp\left[ -\frac{1}{2}#1^\top #2^{-1} #1 \right]}}

% VECTORS AND MATRICES USED IN THIS PAPER
\newcommand{\Normal}{\ensuremath{\mathcal{N}}}
\newcommand{\mA}{\ensuremath{\bvec{A}}}
\newcommand{\mC}{\ensuremath{\bvec{C}}}
\newcommand{\mS}{\ensuremath{\bvec{\Sigma}}}
\newcommand{\mL}{\ensuremath{\bvec{\Lambda}}}
\newcommand{\vw}{\ensuremath{\bvec{w}}}
\newcommand{\vy}{\ensuremath{\bvec{y}}}
\newcommand{\vt}{\ensuremath{\bvec{\theta}}}
\newcommand{\vm}{\ensuremath{\bvec{\mu}(\bvec{\theta})}}
\newcommand{\vre}{\ensuremath{\bvec{r}}}
\newcommand{\vh}{\ensuremath{\bvec{h}}}
\newcommand{\vk}{\ensuremath{\bvec{k}}}

% typography obsessions
\setlength{\parindent}{3.0ex}

\begin{document}\raggedbottom\sloppy\sloppypar\frenchspacing

\title{%
Robust characterization and detection of multi-planet radial velocity
exoplanet systems
Characterization of radial velocity exoplanets in multi-planet systems using
gradient-based inference
}

\author[0000-0002-9328-5652]{Daniel Foreman-Mackey}
\affil{Center for Computational Astrophysics, Flatiron Institute, New York, NY}

\author{others}

\begin{abstract}

RVs are rad.
High-dimensional MCMC is Hard.

\end{abstract}

\keywords{%
methods: data analysis ---
methods: statistical
}

\section{Introduction}

Many new high-precision spectrographs are coming online (list some examples)
with the explicit goal of detecting and characterizing exoplanet systems.
In many of these cases, these exoplanets will live in multi-planet systems
with more than one detectable Keplerian signal in the radial velocity time
series.
It is standard practice to characterize these signals in a Bayesian framework
(e.g.~radvel) where inferences are made about the properties of the system
using a probabilistic model and Markov chain Monte Carlo (MCMC).
Since MCMC is an iterative method, it must be run until convergence to obtain
robust results.
The number of computations required to reach convergence for multi-planet
systems using standard methods (e.g.~emcee) is prohibitively large since the
computational cost per independent sample scales as a high power of the number
of dimensions.

Instead, if we can compute the gradient of the log probability function with
respect to the parameters of the model, we can use gradient-based methods in
the Hamiltonian Monte Carlo (HMC) class of algorithms.
These methods generally scale better to large numbers of parameters---and we
will demonstrate that the same is true for radial velocity exoplanets.

\section{Keplerian radial velocity exoplanet models \& their gradients}

The main step in this model that is not trivially differentiable in a standard
automatic differentiation framework is the solver for Kepler's equation.
We must solve Kepler's equation
\begin{eqnarray}
M = E - e\,\sin E
\end{eqnarray}
for the eccentric anomaly $E$ given some mean anomaly $M$ and eccentricity
$e$.
This can be solved numerically using standard methods (cite), but if we want
to use gradient-based methods, we must also be able to compute the gradients
$\dd E / \dd M$ and $\dd E / \dd e$.
These can be calculated by implicitly differentiating the above equation to
find
\begin{eqnarray}
\dd M &=& (1 - e\,\cos E)\,\dd E - \sin E \, \dd e \\
\dd E &=& \frac{1}{1 - e\,\cos E}\,\dd M + \frac{\sin E}{1 - e\,\cos E}\,\dd e
\end{eqnarray}
giving the required partial derivatives
\begin{eqnarray}
\frac{\dd E}{\dd M} &=& \frac{1}{1 - e\,\cos E} \\
\frac{\dd E}{\dd e} &=& \frac{\sin E}{1 - e\,\cos E} \quad.
\end{eqnarray}

\section{Outline}

Radvel \citep{Fulton:2018}


\bibliography{rvhmc}


\end{document}
